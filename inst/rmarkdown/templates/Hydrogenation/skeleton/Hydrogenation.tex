\documentclass[]{article}
\usepackage{lmodern}
\usepackage{amssymb,amsmath}
\usepackage{ifxetex,ifluatex}
\usepackage{fixltx2e} % provides \textsubscript
\ifnum 0\ifxetex 1\fi\ifluatex 1\fi=0 % if pdftex
  \usepackage[T1]{fontenc}
  \usepackage[utf8]{inputenc}
\else % if luatex or xelatex
  \ifxetex
    \usepackage{mathspec}
  \else
    \usepackage{fontspec}
  \fi
  \defaultfontfeatures{Ligatures=TeX,Scale=MatchLowercase}
\fi
% use upquote if available, for straight quotes in verbatim environments
\IfFileExists{upquote.sty}{\usepackage{upquote}}{}
% use microtype if available
\IfFileExists{microtype.sty}{%
\usepackage{microtype}
\UseMicrotypeSet[protrusion]{basicmath} % disable protrusion for tt fonts
}{}
\usepackage[margin=1in]{geometry}
\usepackage{hyperref}
\hypersetup{unicode=true,
            pdftitle={Insert Title},
            pdfauthor={Enter Name},
            pdfborder={0 0 0},
            breaklinks=true}
\urlstyle{same}  % don't use monospace font for urls
\usepackage{graphicx,grffile}
\makeatletter
\def\maxwidth{\ifdim\Gin@nat@width>\linewidth\linewidth\else\Gin@nat@width\fi}
\def\maxheight{\ifdim\Gin@nat@height>\textheight\textheight\else\Gin@nat@height\fi}
\makeatother
% Scale images if necessary, so that they will not overflow the page
% margins by default, and it is still possible to overwrite the defaults
% using explicit options in \includegraphics[width, height, ...]{}
\setkeys{Gin}{width=\maxwidth,height=\maxheight,keepaspectratio}
\IfFileExists{parskip.sty}{%
\usepackage{parskip}
}{% else
\setlength{\parindent}{0pt}
\setlength{\parskip}{6pt plus 2pt minus 1pt}
}
\setlength{\emergencystretch}{3em}  % prevent overfull lines
\providecommand{\tightlist}{%
  \setlength{\itemsep}{0pt}\setlength{\parskip}{0pt}}
\setcounter{secnumdepth}{0}
% Redefines (sub)paragraphs to behave more like sections
\ifx\paragraph\undefined\else
\let\oldparagraph\paragraph
\renewcommand{\paragraph}[1]{\oldparagraph{#1}\mbox{}}
\fi
\ifx\subparagraph\undefined\else
\let\oldsubparagraph\subparagraph
\renewcommand{\subparagraph}[1]{\oldsubparagraph{#1}\mbox{}}
\fi
\usepackage[labelfont=bf, margin=2in]{caption}
\usepackage{floatrow}
\usepackage{tabularx}
\floatsetup[table]{capposition=top}

%%% Use protect on footnotes to avoid problems with footnotes in titles
\let\rmarkdownfootnote\footnote%
\def\footnote{\protect\rmarkdownfootnote}

%%% Change title format to be more compact
\usepackage{titling}

% Create subtitle command for use in maketitle
\newcommand{\subtitle}[1]{
  \posttitle{
    \begin{center}\large#1\end{center}
    }
}

\setlength{\droptitle}{-2em}
  \title{Insert Title}
  \pretitle{\vspace{\droptitle}\centering\huge}
  \posttitle{\par}
  \author{Enter Name}
  \preauthor{\centering\large\emph}
  \postauthor{\par}
  \predate{\centering\large\emph}
  \postdate{\par}
  \date{Insert Date}

\begin{document}
\maketitle

\subsection{Abstract}\label{abstract}

A brief description of the purpose, procedure, and conclusions.

\subsection{Calculations}\label{calculations}

Insert a figure (scan) of your calculations from your notebook. To
insert a figure, with a caption see the iron analysis lab or see:
\url{http://rmarkdown.rstudio.com/authoring_basics.html}

The calculations to include are:

\begin{enumerate}
\def\labelenumi{\arabic{enumi}.}
\tightlist
\item
  Percentage of each fatty acid
\item
  Total Mass of each triglyceride
\item
  Molar Mass of each triglyceride
\item
  Moles of each triglyceride
\item
  Moles C=C double bonds
\item
  Moles H2
\item
  Conversion from moles H2 to L
\item
  \% Efficiency Calculations
\item
  Average (if using a program, write what program was used); If using R
  to calulcate see:
  \url{http://www.r-tutor.com/elementary-statistics/numerical-measures/mean}
\item
  Standard Deviation (if using a program, write what program was used);
  If using R to calculate see:
  \url{http://www.r-tutor.com/elementary-statistics/numerical-measures/standard-deviation}
\item
  95\% Confidence Interval
\end{enumerate}

\subsection{Results}\label{results}

Text that describes both your observations as well as commentary on your
GC/FAME results and your hydrogenation results. Look at Iron\_Analysis
markdown file on how to include a picture with a caption in your report,
the figure should be of your FAME chromatogram.

\subsection{Discussion}\label{discussion}

Analysis of your results. See page 9.10 in your lab manual for more
detailed information on what this section should include.

FYI: There is a built in spell check for your R markdown file under the
Edit menu in Rstudio. you might want to use that before knitting to PDF
: )

\begin{table}[ht]
\centering
\parbox{2.7in}{\caption{A lovely caption for the table}} 
\begin{tabular}{|p{0.7in}|p{0.7in}|p{0.7in}|p{0.7in}|p{0.7in}|p{0.9in}|p{0.7in}|}
  \hline
\textbf{FAME Label} & \textbf{ID Label} & \textbf{Rt Label (units)} & \textbf{Area Label (units)} & \textbf{Mass Percent Label} & \textbf{ClassAverage +/- 95\% CI (units)} & \textbf{Published Mass Percent} \\ 
  \hline
16:0 & Acid A &       4 &     100 &       7 & 10+/-1 & 7.9-10.2 \\ 
   \hline
18:0 & Acid B &       6 &     200 &      13 & 20 +/- 1 & 4.8-6.1 \\ 
   \hline
18:1 & Acid C &       8 &     300 &      20 & 30 +/- 1 & 35.9-42.3 \\ 
   \hline
18:2 & Acid D &      10 &     400 &      27 & 40 +/- 1 & 41.5-47.9 \\ 
   \hline
18:3 & Acid E &      12 &     500 &      33 & 50 +/- 1 & 0.3-0.4 \\ 
   \hline
\end{tabular}
\end{table}


\end{document}
